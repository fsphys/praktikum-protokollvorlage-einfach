\documentclass[11pt]{scrreprt}
\usepackage[left=25mm,right=25mm,top=25mm,bottom=25mm]{geometry}
\usepackage[utf8]{inputenc}
\usepackage[ngerman]{babel}
\usepackage{float}
\usepackage{graphicx}
\usepackage{textcomp, gensymb}
\usepackage{caption}
\usepackage{xcolor}
\usepackage{url}
\usepackage{hyperref}
\usepackage{varioref}
\usepackage{mathtools}
\usepackage{amssymb}
\usepackage{siunitx}
\usepackage[version=4]{mhchem}
\usepackage{cleveref}
\usepackage{pdfpages}
\usepackage{tikz}
% Schönere Tabellen
\usepackage{booktabs}
% Plotten, Tabellen und vieles mehr
\usepackage{pgfplots}
\pgfplotsset{compat=1.16}


\begin{document}

% Verhindert die Seitennummerierung auf der Titelseite
\pagenumbering{gobble}

%--------Deckblatt Versuchsprotokoll----------%
% Der input-Befehl lädt den LaTeX-Code aus der angegeben Datei und fügt ihn an dieser Stelle ein.
% Hier müsst ihr dafür sorgen, dass die entsprechend nicht genutzte Version mit einem % auskommentiert ist
\input{chapters/DeckblattP1P2}
%\input{chapters/DeckblattP3P4}
%---------------------------------------------%

\newpage
\pagenumbering{arabic} % Ab hier beginnt die Seitennummerierung
\setcounter{page}{1}   % mit Seitenzahl 1. Dies stimmt mit der Seitennummerierung der Aufgabenblätter überein.

%--------Aufgabenblatt----------%
% kann so dem Inhaltsverzeichnis hinzugefügt werden
%\addcontentsline{toc}{chapter}{Aufgabenblatt}

% Im P3&P4 gibt es teilweise keine Aufgabenblatt mehr, sodass dieser Abschnitt auskommentiert werden kann.
\includepdf[pages=-]{include/Aufgabenblatt.pdf} 
% Die Datei Aufgabenblatt.pdf mit dem Inhalt "Test-Test-Test-Test" muss durch das aktuelle Aufgabenblatt des Versuchs ausgetauscht werden.
%-------------------------------%

%--------Inhaltsverzeichnis, Tabellen- und Abbildungsverzeichnis----------%
\setcounter{chapter}{0} 
% Bei "\setcounter{chapter}{-1}" erhält die Einführung die Kapitelnummer "0";
% Dies kann praktisch sein, um die Kapitelnummer mit den Aufgabennummern der Versuche gleich zu halten.
% Bei "\setcounter{chapter}{0}" startet das Inhaltsverzeichnis bei "1".
\tableofcontents % Erstellt ein Inhaltsverzeichnis
%\vspace{50px}   
%\listoffigures  % Erstellt ein Abbildungsverzeichnis. Dies wird von manchen Tutor:innen gefordert.
%\vspace{50px}   
%\listoftables  % Erstellt ein Tabellenverzeichnis. Dies wird von manchen Tutor:innen gefordert.
\pagebreak
%-------------------------------------------------------------------------%

%--------Inhalt der Kapitel----------%
\addcontentsline{toc}{chapter}{Einführung}
\chapter*{Einführung}
% Hier werden unter anderem die Ziele des Versuchs beschrieben

\input{chapters/01}

\input{chapters/02}

\input{chapters/03}

\input{chapters/04}
%------------------------------------%

%--------Quellen----------%
\input{chapters/Quellen}
%-------------------------%

%--------Anhang (Messprotokoll)----------%
\newpage
\addcontentsline{toc}{chapter}{Messprotokoll}
\includepdf[pages=-]{include/Messprotokoll.pdf} 
% Die Datei Messprotokoll.pdf mit dem Inhalt "Test-Test-Test-Test" muss durch das, während des Versuchs erstelltem Messprotokoll, ausgetauscht werden.
%----------------------------------------%

\end{document}